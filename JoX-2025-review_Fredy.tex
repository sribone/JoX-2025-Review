\documentclass[12pt,a4paper, margin=2cm]{article}
\usepackage[utf8]{inputenc}
\usepackage[T1]{fontenc}
\usepackage{amsmath}
\usepackage{amssymb}
\usepackage{graphicx}
\usepackage[english]{babel}
\usepackage{geometry}
\geometry{margin=2cm}
\usepackage{subcaption}
\usepackage{array}
\usepackage{tabularray}
\usepackage{longtable}
\usepackage{pdflscape}
\usepackage{color}
\usepackage{multicol}
\usepackage{setspace}

% Do not compile the figure while mantaining the figure box
%\usepackage[allfiguresdraft]{draftfigure}

% To center content in table cells
\newcolumntype{C}[1]{>{\centering\arraybackslash}m{#1}}

\doublespacing

\begin{document}
	\begin{center}
		\textbf{Experimental and in silico aproaches to study CES substrate selectivity}
		\vspace{0.25cm}

		Sergio R. Ribone and Mario A. Quevedo 

		\vspace{0.25cm}

		Universidad Nacional de C\'ordoba. Facultad de Ciencias Qu\'imicas. Departamento de Ciencias Farmac\'euticas. Consejo Nacional de Investigaciones Cient\'ificas y T\'ecnicas (CONICET), Unidad de Inve
stigaci\'on y Desarrollo en Tecnolog\'ia Farmac\'eutica (UNITEFA), C\'ordoba (X5000HUA), Argentina.\\

		\vspace{0.25cm}

		*sribone@unc.edu.ar


	\end{center}
	
    \newpage

	\begin{center}
		\textbf{Abstract}    
    \end{center}

%  UPDATED by Fredy in revision2:  Human carboxylesterases (CES) are enzymes that play a central role in the metabolism and biotransformation of diverse substances, where the two most relevant  isoforms, CES1 and CES2, catalyze the hydrolysis of numerous approved drugs and prodrugs. Understanding CES isoform substrate specificity is crucial for multiple research areas: 1) the design of prodrugs with optimized site-specific bioactivation, 2) the development of specific inhibitors for CES1 and CES2 and 3) the design of fluorescent probes that specifically target CES1 and CES2 in complex biological systems. Various experimental and computational methodologies have been developed to quantify CES kinetic parameters (k\textsubscript{cat} and K\textsubscript{M}) for different substrates.This review will focus in the recent advancements in these methodologies to study substrate selectivity between CES1 and CES2. Experimental measurements commonly use recombinant CES or human tissue microsomes as enzyme sources. The quantificationmethods used were spectrophotomety (UV and fluorescence) and mass spectrometry, where in most cases is previously couple with separation chromatographic methodologies in order to increase accuracy. Computational approaches are typically divided into two categories: 1) the modeling of substrate:CES recognition and affinity (molecular docking, molecular dynamic simulation and free-energy of binding) and 2) the modeling of  the substrate:CES hydrolysis (hybrid QM/MM simulation). Both approaches have demonstrated high accuracy in the explanation of experimental results. The advantages of integrating experimental and computational techniques are evident in several studies that clarify the structural and mechanistic basis of CES substrate selectivity. Given the biological relevance of CES-mediated catalysis, this review aims to provide a concise resource for continued exploration of CES isoform specificity and its implications for drug, prodrug and fluorescent proves design.
	 
	Human carboxylesterases (CES) are enzymes central to the metabolism and biotransformation of diverse endogenous and exogenous substances. The two most relevant isoforms, CES1 and CES2, are crucial in clinical pharmacotherapy as they catalyze the hydrolysis of numerous approved drugs and prodrugs. Elucidating the structural basis of CES isoform substrate specificity is essential not only for understanding and anticipating the biological fate of administered drugs, but also for designing prodrugs with optimized site-specific bioactivation. Additionally, the development of specific inhibitors for CES1 and CES2 is being pursued, while fluorescent probes targeting these isoforms represent increasingly important tools for studying complex biological systems.
	
	Both experimental and computational methodologies have been used to explore the mechanistic and thermodynamic properties of CES-mediated catalysis. Experimental designs commonly employ recombinant CES or human tissue microsomes as enzyme sources, utilizing quantification methods such as spectrophotometry (UV and fluorescence) and mass spectrometry. Computational approaches fall into two categories: 1) modeling substrate:CES recognition and affinity (molecular docking, molecular dynamics simulation, and free-energy binding calculations), and 2) modeling substrate:CES reaction coordinates (hybrid QM/MM simulations).
	
	This review focuses on the advantages of integrating experimental and computational techniques, along with recent advancements, providing a concise resource for continued exploration of CES isoform specificity and its implications for drug, prodrug, and fluorescent probe design.
	
	\vspace{1.5cm}
	
	\textbf{Keywords:} Carboxylesterases; CES1; CES2; substrate; selectivity; HPLC; LC/MS; Docking; MD simulation; QM/MM

    \newpage

\section{Introduction}

Human carboxylesterases (CES, EC 3.1.1.1) are enzymes belonging to the serine hydrolase family, responsible for the metabolism and biotransformation of diverse substances containing ester, thioester, amide, carbonate, and carbamate moieties\cite{Wang2018a,Fukami2012,Hosokawa2008}. CES are classified based on amino acid sequence homology into five isoforms (CES1–CES5), among which CES1 and CES2 constitute the most clinically relevant isoforms\cite{CaseyLaizure2013,Dai2020,Xu2016}.

Although CES1 and CES2 share 47\% protein sequence identity, they exhibit distinct substrate specificity profiles and tissue distributions. CES1 is primarily expressed in the liver, whereas CES2 is predominantly found in intestinal tissue\cite{Wang2018a,CaseyLaizure2013,Hosokawa2008,Di2019}. It has been generally accepted that the substrate binding and hydrolysis selectivity of each isoform is primarily determined by the size of the acyl and alkyl moieties in the substrate molecular structure\cite{Di2019,CaseyLaizure2013,Xu2016,Fukami2015}. CES1 preferentially catalyzes the hydrolysis of substrates with smaller alkyl than acyl groups, as observed with drugs and prodrugs like clopidogrel, oseltamivir, and meperidine\cite{Zhang2014,Fukami2015,Fukami2012}. Conversely, CES2 preferentially hydrolyzes compounds with larger alkyl than acyl groups, including haloperidol, procaine, and flutamide\cite{Wang2018a,Fukami2015,Hosokawa2008}.

Despite this general trend, several exceptions to the substrate selectivity rule have been reported. For instance, drugs and prodrugs such as irinotecan, propanil, oxybutynin, and procaine, with different proportions of alkyl and acyl group sizes, are metabolized by both CES isoforms with similar efficiency\cite{Fukami2015,Honda2021}. A widely reported exception is heroin, which is mainly metabolized by CES1 despite having a larger alkyl moiety relative to the acyl group\cite{Bencharit2003}. These and other contradictory reports strongly suggest that CES substrate selectivity is governed by subtle atomistic details that drive catalytic efficiency. This is further supported by the differential CES1 and CES2-mediated biotransformation rates of \textit{cis} and \textit{trans} isomers of permethrin. Despite having identical acyl and alkyl group sizes, trans-permethrin is efficiently hydrolyzed by both CES isoforms, while cis-permethrin is hydrolyzed mainly by CES2\cite{Yang2009}. A similar phenomenon was reported for the different hydrolysis patterns of eight cypermethrin and four fenvalerate stereoisomers\cite{Nishi2006}.


\begin{center}
	\textbf{Acá porqué no poner una figura con el especto de selectividad?}
\end{center}

Understanding the structural determinants of substrate specificity among CES isoforms is crucial for optimizing drug therapy and developing targeted pharmacotherapeutic strategies, given CES enzymes' critical role in physiological catalysis.

\begin{center}
	\textbf{Acá me parece que falta un párrafo que haga referencia a la capacidad de anticipar el perfil metabólico de un fármaco relacionado con el subtipo CES}
\end{center}

CES subtype-mediated bioactivation has enabled the design of prodrugs with optimized pharmacotherapeutic profiles. This approach prevents undesirable early biotransformation while enabling site-specific bioactivation by specific CES isoforms\cite{Pratt2013,Fukami2012}. Currently, this strategy is being evaluated in a clinical trial examining CES1-expressing allogeneic neural stem cells combined with irinotecan for high-grade brain gliomas. The hypothesis is that intracranial administration of genetically modified neural stem cells enhances tumor sensitivity to irinotecan through site-specific bioactivation\cite{Portnow2016}. Similar targeting approaches using immortalized adipose-derived stem cells engineered to express carboxylesterase 2 have shown promise for castration-resistant prostate cancer treatment\cite{kim2025combination}, demonstrating the therapeutic potential of both endogenous CES levels and stem cell-engineered enzyme delivery\cite{sher2024cellular}. Amide-derived gemcitabine prodrugs exemplify the exploitation of endogenous CES levels for solid tumor treatment. These prodrugs are selectively bioactivated by CES2, which is overexpressed in certain cancer types\cite{Pratt2013}. Given this CES2-mediated targeting potential, CES2 expression levels have been proposed as a prognostic biomarker for breast cancer\cite{qu2023prognostic}.

Specific inhibitors for CES1 and CES2 offer promising therapeutic potential for managing metabolic diseases\cite{wang2024pyrazolone, wang2024increased}. For example, the CES1 inhibitor GR148672X has been explored for treating hypertriglyceridemia, obesity, and atherosclerosis\cite{Bachovchin2012}. However, CES1 inhibition can affect the metabolism of other drugs, resulting in clinically significant drug-drug interactions. Several clinical trials are investigating this effect, including one evaluating how concomitant cannabidiol treatment (a CES1 inhibitor) affects methylphenidate metabolism (a CES1 substrate)\cite{Markowitz2021}.

The development of fluorescent biological probes for selective CES1 and CES2 activity detection represents a growing area of interest in medical imaging\cite{Zhang2021,Elkhanoufi2022,Dai2021,Dai2020,Jia2021}. Selective \textit{in vivo} enzyme activity imaging has emerged as a powerful method for studying biological systems by providing real-time, noninvasive monitoring within living organisms. Fluorescent probes specifically targeting CES1 and CES2 offer a promising strategy for visualizing their activity in complex biological systems\cite{wang2025ratiometric, zhou2025strategies, iqbal2024real}. \textbf{[FREDY: Siento que faltan mostrar casos específicos como en los párrafos anteriores]}

Enzymology has long been foundational for studying enzyme structure and function, advancing our understanding of intermediary metabolism, molecular biology, and cellular signaling\cite{Punekar2025}. Early enzymology focused on experimental techniques analyzing both catalytic properties and molecular specificity. The first approach studies the thermodynamics and kinetics of enzymatic reactions, measuring reaction free energy and activation energy. The second examines molecular specificity for enzymes, including not only substrates but any molecule meeting the enzyme's specificity criteria, such as potential inhibitors\cite{Punekar2025}.

Due to enzyme complexity and the challenges of studying biomolecular reactions, many mechanisms remain unclear. Computational enzymology - the study of enzymes and their reaction mechanisms through molecular modeling and simulation\cite{nam2024perspectives, VanDerKamp2013} - uniquely enables investigation of biomolecular dynamic behavior and reactions at atomic resolution. This approach addresses unresolved issues by complementing and interpreting experimental findings\cite{VanDerKamp2013,Lonsdale2010,Lodola2012}. 
Since the pioneering 1976 work of Warshel and Levitt (2013 Nobel laureates), computational enzymology has rapidly evolved through close collaboration between experimental and computational enzymologists, enhancing our ability to explain and interpret experimental data\cite{cui2025approaches, VanDerKamp2013,Lonsdale2010,Lodola2012}.

A major advantage for experimental and computational enzymology is the availability of crystallographic structures. Human CES crystallographic structures reveal three main functional domains: the catalytic domain, the $\alpha/\beta$ domain, and the regulatory domain\cite{Yao2018,Vistoli2010a}. In both CES isoforms, the catalytic site resides within the catalytic domain and contains the classical Ser-His-Glu triad characteristic of serine hydrolases (Ser221-His467-Glu354 and Ser228-His457-Glu345 for CES1 and CES2, respectively)\cite{Yao2018,Vistoli2010a}. Additionally, an oxyanion hole is present within the catalytic site, formed by residues Gly142-Gly143-Ala222 and Gly149-Ala150-Ala229 in CES1 and CES2, respectively\cite{Yao2018,Vistoli2010a}.


\begin{center}
	\textbf{Porqué no una imágen 3D de las enzimas?}
\end{center}

The general hydrolysis mechanism of esterases, including CES, involves two consecutive reaction steps (Figure~\ref{fig:CES_mechanism}). First, during the acylation step, the hydroxyl group of the catalytic serine attacks the carbonyl carbon of the substrate, forming an acylated serine intermediate and releasing the alkyl group. Second, during deacylation, a water molecule attacks the carbonyl carbon of the acylated serine, yielding the carboxylic acid portion of the substrate and regenerating the free serine residue, allowing a new catalytic cycle to begin (Figure~\ref{fig:CES_mechanism})\cite{Hosokawa2008,Yao2018,Ribone2025}.


\begin{center}
    \textbf{Figure~\ref{fig:CES_mechanism}}
\end{center}

Building on the symbiotic relationship between experimental and computational enzymology, this review focuses on recent methodological advancements for studying substrate selectivity between the two major human carboxylesterase isoforms: CES1 and CES2. The first section discusses experimental enzymology approaches, covering enzyme sources and methodologies for calculating kinetic parameters. The second section examines molecular modeling techniques used to study substrate affinity and hydrolytic properties for both CES isoforms. Finally, the third section reviews current efforts combining experimental and computational studies to provide comprehensive analyses of CES-substrate selectivity.

\section{Part I: Experimental enzymology}

\subsection{Determination of kinetic parameters}

During enzymatic property studies, important kinetic parameters are determined using the fundamental Michaelis-Menten equation\cite{Punekar2025,Michaelis1913}. The first parameter is the \textbf{dissociation constant}, also known as the Michaelis constant (K\textsubscript{M}), which reflects enzyme-substrate affinity. Lower K\textsubscript{M} values indicate stronger binding within the enzyme-substrate complex\cite{Punekar2025,Nishi2006,Fukami2012,Ribone2025}. The second parameter is the \textbf{catalytic turnover}, determined by the catalytic constant (\textit{k\textsubscript{cat}}), which represents the number of catalytic cycles the enzyme completes per unit time when fully saturated with substrate. Higher \textit{k\textsubscript{cat}} values indicate greater substrate turnover and more efficient metabolism\cite{Punekar2025,Nishi2006,Fukami2012,Ribone2025}.

The ratio of these kinetic parameters (\textit{k\textsubscript{cat}}/K\textsubscript{M}) is the \textbf{specificity constant}, reflecting the enzyme's ability to discriminate between substrates. Higher \textit{k\textsubscript{cat}}/K\textsubscript{M} ratios indicate high substrate affinity (low K\textsubscript{M}) and high catalytic rates. Therefore, the specificity constant serves as a measure of \textbf{catalytic efficiency}, determining whether a given molecule is a good or poor substrate for the enzyme\cite{Punekar2025,Ribone2025}.

To obtain kinetic parameters that quantify substrate selectivity, experimental enzymology measures the progress of enzyme-catalyzed reactions. Like other chemical reactions, enzyme-mediated substrate hydrolysis can be monitored by measuring either product formation or substrate consumption. Adequate detection methods for these processes are essential for successful enzyme assays\cite{Punekar2025}. Table~\ref{table:data} summarizes important information about enzyme sources and analytical methods used to explore CES kinetic parameters with different reported substrates.


\begin{center}
	\textbf{Table~\ref{table:data}}
\end{center}

\subsection{Enzyme sources}

This section outlines the different CES enzyme sources used in reported experimental setups for calculating kinetic parameters. The two primary sources are \textit{ex vivo} tissues (including purified human tissues and human microsomes) and pure recombinant enzymes (Table~\ref{table:data}).

\begin{center}
\textbf{[FREDY: te parece funcional crear una subsub sección. Porqué no poner todo en en la sección de "Enzyme sources" para que queden secciones mas unificadas?]}
\end{center}

\subsubsection{Obtention of CES from \textit{ex vivo} tissues}

DEJO ACA

Since the early 2000's, the enzymatic studies of CES1 and CES2 substrate hydrolysis have relied on purified isoforms sourced from human liver. 
In several studies, human liver tissue is processed by homogenization and centrifugation, followed by separation of CES1 and CES2 isoforms using
chromatographic columns\cite{Humerickhouse2000,Sanghani2004,Sun2004}. Using this method, it was determined that the produg irinotecan, along with 
other metabolites, is primarily bioactivated by human CES2 isoform (Table~\ref{table:data}) \cite{Humerickhouse2000,Sanghani2004}. 



Another \textit{ex-vivo} source of CES is human tissue microsomes, which are small vesicles derived from fragmented cell membranes, mainly endoplasmic 
reticulum. These human microsomes can be obtained by differential centrifugation of the corresponding tissue or purchased from different biological supply companies. Human liver microsomes (HLM) are commonly used as enzyme source for measuring metabolic stability, as they contain key metabolizing enzymes, including CES. As mentioned in the introduction, CES1 is predominantly expressed in the liver, making HLM a valuable source for studies of CES1 substrate selectivity. Following the previous mentioned tissue distribution patterns, human intestines microsomes (HIM) have been used as a source of CES2 in selectivity assays. Using this approach, studies have shown that the antiviral produgs oseltamivir and temocapril are preferentially activated by CES1 (HLM) over CES2 (HIM)\cite{Shi2006,Imai2006}.

\subsubsection{Recombinant enzyme}

Shortly after the use of human tissue as enzyme source, it became evident that a more purified source of both CES isoforms were necessary to perform
accurate selectivity enzymatic experiments with different substrates. Morton and Potter developed a method for cloning and expressing CES
using baculovirus to infect \textit{Spodoptera frugiperda} insect cells \cite{Morton2000}. This technique enabled the production of recombinant CES1
and CES2, which were then used to measure the enzymatic parameters of several substrates (Table\ref{tab1}). The hydrolysis of pyrethroids by human
CES1 and CES2 have been studied using this recombinant enzyme source \cite{Nishi2006,Ross2006}. These reports showed that the CES isoforms
displayed different enantio/diastereo-selectivities for these substrates and also that fluorescent derivatives can be used to evaluate hydrolysis 
activity/selectivity among CES1 and CES2\cite{Nishi2006,Ross2006}.
Additionally, the hydrolysis specificity of 
CES1 and CES2 for drugs of abuse, such as heroin and cocaine\cite{Hatfield2010}, as well as other drugs and prodrugs, have also been studied using
recombinant enzymes (Table\ref{tab1})\cite{Sato2012,Higuchi2013,Fukami2015}.

As recombinant CES enzymes became commercially available from multiple suppliers, research groups were able to continue exploring the metabolism selectivity of CES isoforms across different substrates. For example, studies on the angiotensin-converting enzyme inhibitors enalapril and ramipril showed that both drugs were selectively hydrolyzed by CES1\cite{Thomsen2014}. Recombinant CES enzymes have also been used to determine the predominant role of CES isoforms in the human bioactivation of prodrugs such as sacubitril\cite{Shi2016} and anordrin\cite{Jiang2018}. Additionally, recombinant CES enzyme have facilitated preclinical evaluations of the bioactivation rates for structurally designed prodrugs, including atorvastatin\cite{Mizoi2016,
Takahashi2025}, indomethacin\cite{Takahashi2018,Takahashi2020} and haloperidol\cite{Takahashi2019}.

\subsection{Reported analytical methods}

As mentioned earlier, obtaining accurate kinetic parameters requires a reliable analytical method to quantify the progress of the 
enzyme-catalyzed reaction. In this section, the different analytical methods used for the determination of the described kinetic parameters
for different substrates and both CES isoforms will be described.The analytical methods are classify in: ultraviolet (UV) spectrophotometry,
fluorescence spectrophotometry and mass spectrometry.

\subsubsection{UV spectrophotometry}
 
UV spectrophotometric methods have been widely used to quantitatively determine total CES activity, by measuring the absorbance of \textit{p}-nitrophenol, produced through the hydrolysis of \textit{p}-nitrophenyl acetate (pNPA), at 405 nm (Figure~\ref{fig:pNPA_hydrolysis})\cite{Wadkins2001,Hatfield2010,Boonyuen2015}. This approach was applied to the functional characterization of recombinant human CES expressed in \textit{E. coli} as an alternative method for obtaining the human enzyme\cite{Boonyuen2015}.

\begin{center}
    \textbf{Figure~\ref{fig:pNPA_hydrolysis}}
\end{center}

The absorbance properties of \textit{p}-nitrophenol has also been used to evaluate the kinetic parameters of various \textit{p}-nitrophenyl 
esters derivatives with CES1 and CES2\cite{Hatfield2010}. The authors observed a correlation between the affinity constant (K\textsubscript{M}) and the calculated water/octanol partition coefficients (clogP) values, concluding that the affinity of the substrates for both CES isoforms is directly related to their lipophilicity properties\cite{Hatfield2010}. In addition, kinetic data for naphthyl esters derivatives were obtained by measuring the formation of naphthol at 230nm
\cite{Wadkins2001}.

This method is simple and rapid, but it has several disadvantages. One key issue is the potential interference between substrates and products, which complicates experiments in complex biological systems. Additionally, the method requires higher amounts of enzyme, as it is performed in a UV cuvette
with a total volume of 1 ml\cite{Boonyuen2015}.

In order to be able to separate substrate from products for a more precise quantification and higher reproducibility, an UV-detector can be couple after a chromatographic separation method. This strategy is known as HPLC-UV and it has been widely used to study the CES kinetic parameters of various therapeutic drugs and prodrugs, as most ligands exhibit absorbance in the ultraviolet spectrum\cite{Fukami2015,Watanabe2009,Kobayashi2012,Higuchi2013,Hatfield2010,Takahashi2018,Takahashi2019,Takahashi2020,Takahashi2021,Takahashi2025}.

To investigate substrate specificity among CES isoforms, a study was conducted on 13 compounds, including clopidogrel, clofibrate, oseltamivir,
mycophenolate mofetil, procaine and temocapril, among others (Table~\ref{tab1}), using HPLC with specific conditions for each metabolite
(e.g., mobile phase, column and UV wavelength)\cite{Fukami2015}. This research group also applied the HPLC-UV method to study the
kinetics of other drugs, such as flutamide\cite{Watanabe2009,Kobayashi2012}, prilocaine and lidocaine\cite{Higuchi2013}. Based on the structural characteristics of these compounds, the authors proposed the already discussed general substrate selectivity pattern for CES: CES1 preferentially hydrolyzes ligands with smaller alkyl than acyl moiety (e.g., clofibrate, lidocaine, temocapril), while CES2 favors those with larger alkyl than acyl groups (e.g., flutamide, procaine).

The metabolism of the abuse drugs cocaine and heroine by CES1 and CES2 was also studied using this methodology\cite{Hatfield2010}. For cocaine, hydrolysis was monitored by quantifying the formation of benzoylecgonine and benzoic acid at 235nm. After incubation with both CES isoforms, the results showed that cocaine was exclusively metabolized by CES2, producing benzoic acid and ecgonine methyl ester. The second potential metabolic pathway, forming benzoylecgonine and methanol, was not detected\cite{Hatfield2010}.

For heroine, hydrolysis was assess by monitoring the formation of 6-acetylmorphine, also at 235nm. Both CES isoforms were found to hydrolyze
heroin, with CES1 exhibiting a higher catalytic efficiency (\textit{k\textsubscript{cat}}/K\textsubscript{M}) compared to CES2\cite{Hatfield2010}.
This result represents an exception to the general CES1 substrate specificity, as was already mentioned in the introduction, as heroin has a larger size alkyl group than acyl moiety in its structure.

Takahashi et al. have been reported several studies where the investigated substrates were structurally diverse indomethacin-derived prodrugs. The formation of indomethacin, monitored at 254 nm, was used as the analytical marker to determine the kinetic parameters\cite{Takahashi2018,Takahashi2020,Takahashi2021}. By synthesizing prodrugs bearing a variety of alkyl moieties, the authors were able to investigate how different structural features influence the hydrolysis behavior of both CES isoforms. Specifically, they examined: 1) the effect of the steric hindrance on the carbon adjacent to the carbonyl group, 2) the influence of electron density around the carbonyl group and 3) the chiral recognition ability between CES1 and CES2. Overall, the results indicated that the indomethacin prodrugs were mainly hydrolyzed  by CES1, because the drug structure represents the acyl group, which in all the cases was larger than the corresponding alkyl group. However, prodrugs with aryl-containing alkyl moieties  exhibited reduced or even no selectivity between CES isoforms, demonstrating that steric hindrance near the ester carbonyl carbon plays a crucial role in determining metabolic selectivity.

This research group also conducted similar studies by synthesizing prodrugs derivatives of haloperidol\cite{Takahashi2019} and atorvastatin
\cite{Takahashi2025}, using the UV properties of these drugs as detection methods couple with HPLC method.


\subsubsection{Fluorescence spectrophotometry}

In contrast to the previously described analytical method, the fluorescent probe-based approach is not only simple
but also highly selective and sensitive. Known as "off-on" fluorescent probes, they initially exhibited little or no fluorescence ("off"), but the hydrolyzed products release strong fluorescence in presence of the corresponding enzyme ("on"). Several structural different fluorescent probes have been designed to quantify CES activity by detecting changes in fluorescence intensity \cite{Nishi2006,Wang2011,Ding2019}

A very common strategy to quantifying fluorescent proves, either for substrate selectivity measurement or for the screening of enzyme inhibitor,
is using the 96-well flat-bottomed microtiter plates combined with a fluorescent spectrophotometer. In addition to the selectivity and sensitivity
of the fluorescent prove, this methodology posses high practicality, because in everyone of the small wells, total volume of 200$\mu$l, 
is possible to performed an assay at one substrate concentration with the enzyme. With 96 wells is feasible to performed a general assay 
with one substrate at 8 different concentrations, by quadruplicate, with both CES isoforms and a blank without the enzyme in one single run. This methodology has been used to determine the kinetic parameters of pyrethroids-like substrates containing 6-methoxy-2-naphthaldehyde. The fluorescence of this moiety is measured with an excitation wavelength of 330 nm and an emission wavelength of 465 nm\cite{Nishi2006}. In this study, it was observed that the steroisomeric centers of these derivatives presented a differential impact on hydrolysis by the two CES isoforms. Specifically, the presence of an (\textit{R})-enantiomer carbon adjacent to the ester carbonyl carbon resulted in a greater preference for CES2 hydrolysis than CES1\cite{Nishi2006}. These findings highlight the importance of the three-dimensional disposition of the substrate groups within the catalytic site of the enzyme for CES hydrolysis selectivity.

In another study, fluorescein diacetate was used as a substrate to assess CES1 and CES2 selectivity. The hydrolysis of fluorescein diacetate by the CES enzymes releases fluorescein, which is quantified at an excitation wavelength of 483 nm and emission at 525 nm. The results indicated that fluorescein diacetate as a fluorogenic in vitro CES2-selective probe substrate (Figure~\ref{fig2a})\cite{Wang2011}.

\begin{center}
    \textbf{Figure~\ref{fig:fluorescein_hydrolysis}}
\end{center}


The high fluorescence quantum yield and photochemical stability of BODIPY dyes make them excellent candidates for this analytical methodology. As a result, a BODIPY ester was designed as a specific substrate for CES1. The acid product formed after CES hydrolysis was used to measured the kinetic parameters at an excitation wavelength of 505 nm and emission at 560 nm (Figure~\ref{fig2b})\cite{Ding2019}. Furthermore, this probe has also been successfully used for high-throughput screening of CES1 inhibitors using living cells as enzyme sources, demonstrating that this BODIPY derivatives probes were a practical tool for highly selective and sensitive sensing CES1 activities in complex biological systems\cite{Ding2019}.

\begin{center}
    \textbf{Figure~\ref{fig:BODIPY_hydrolysis}}
\end{center}

In these two last studies, the fluorescence was monitored by a fluorescent spectrophotometer after the corresponding separation of substrate and products through chromatographic methods (HPLC).

\subsubsection{Mass spectrometry}

This methodology is the most sensitive and accurate of all the described analytical methods. This is because the mass spectrometer is capable of quantify with high accuracy
the correct mass of every small amount of substance going through the equipment. In this case is also possible to combined this technique with a previous chromatograpic separation, known as liquid chromatography-tandem mass spectrometry (LC/MS), adding also a high degree of reproducibility to this method.

This analytical technique has primarily been used to investigate the metabolic pathway of a diverse set of prodrugs. Interesting results were observed in the hydrolysis study of capecitabine, a carbamate prodrug of 5-fluoruracil used in the treatment of colorectal cancer\cite{Quinney2005}. The study showed that capecitabine is hydrolyzed equally by both CES1 and CES2, suggesting that substrates containing carbamate groups can be metabolized by either isoform, independent of the size of their acyl or alkyl moieties. Other prodrugs studied include plasugrel, which exhibited selectivity for CES2\cite{Williams2008}, sacubitril, selectively activated by CES1\cite{Shi2016}, and anordrin, which showed similar hydrolysis parameters with both CES isoforms\cite{Jiang2018}. This last result is interesting, because in the structure of anordrinthe alkyl moiety is much higher than the acyl group, which it should displayed a CES2 selective hydrolysis\cite{Jiang2018}.


In a more recent study, the hydrolysis characteristics of cocaine by CES1 were re-examined using LC/MS methodology\cite{Yao2018}. This study revealed that cocaine is hydrolyzed by CES1, producing benzoylecgonine and methanol, which contrast with earlier findings using HPLC-UV, where this hydrolysis product was not detected (Figure~\ref{fig:cocaine_hydrolysis})\cite{Hatfield2010}. These results suggest that LC/MS provides a more accurate analytical approach for studying CES selectivity substrate hydrolysis compared to HPLC-UV.


\begin{center}
	\textbf{Figure~\ref{fig:cocaine_hydrolysis}}
\end{center}

This section highlights the importance of experimental enzymology in studying CES substrate selectivity. The gathered information shows that, despite the general structural rules established for substrate selectivity between CES1 and CES2, several exceptions and underlying structural properties remain unclear. Given that CES binding and hydrolysis involve subtle intermolecular interactions, computational enzymology emerges as a crucial tool for analyzing CES substrate selectivity at the atomistic level.

\section{Part II: Computational enzymology}

\subsection{Experimental CES kinetic parameter databases}

In the era of bioinformatics, providing reliable kinetic parameter data and  effective data management systems is essential to support researchers in  retrieving enzymatic information from the vast amount of biological data  published annually, such as those related to CES enzymes. In this context, online databases have become invaluable  tools for granting researchers access to this information. This section will focus on two of the the most well-known and widely used online  databases containing human CES enzymatic parameters for different substrates.

\subsubsection{BRENDA database}

The BRaunschweig ENzyme DAtabase (BRENDA) is the oldest collection of 
enzyme-related data compiled from the scientific literature, established in 1987 at the German National Research Centre for Biotechnology in Braunschweig\cite{Schomburg2017}. Today, the BRENDA website (www.brenda-enzymes.org) is accessed by over 100,000 users each month. The site is very intuitive, allowing users to search by entering a text, such as enzyme name, ligand name, EC class, inhibitors, etc., or through structured-based queries by drawing  substrate/products or ligand substructures\cite{Schomburg2017}.

As of October 2025, the search for CES (EC 3.1.1.1) enzymatic information on BRENDA, website yielded 2079 substrates/products and 67 natural substances. 
Among this data, 849 K\textsubscript{M} values, 550 turnover numbers (\textit{k\textsubscript{cat}}) and 232 \textit{k\textsubscript{cat}}
/K\textsubscript{M} values (catalytic efficiency) can be retrieved from diverse bibliographic sources. In addition, the reaction diagrams and references associated with these substrates are provided. 

However, two main disadvantage were noted using the website: 1) it is not possible to filtrate information based on the organism origin of the enzyme, and 2) the kinetic information can not easily be separate by CES isoforms. Despite this limitations, BRENDA website is a valuable starting point for searching kinetic parameters related to CES1 and CES2 substrate hydrolysis.

An alternative to the web server is working with a python parser tool, (github.com/ Robaina/ BRENDApyrser) that allows the user, after downloading the BRENDA database in a local computer as a text file, to work with the database using Structured Query Language (SQL). This way, is possible to performed different searches or queries through the enzyme code name and process this information to retrieve the kinetic parameters and create a new local database with the desired information. The disadvantage of this strategy is that the user needs to have previous knowledge of python and SQL languages to process and create the local database.


\subsubsection{SABIO-RK database}

SABIO-RK is a manually curated database containing enzymatic biochemical reactions and their kinetic parameters (sabiork.h-its.org/), 
constituting a valuable resource for both  experimental and computational enzymology researchers\cite{Wittig2014}. Data in SABIO-RK are primarily extracted manually from the  literature and stored in a structured and standardized format. The database includes essential data to describe the characteristics of biochemical reactions, the corresponding biological source, kinetic properties and experimental conditions\cite{Wittig2014}.

The regular search for CES name in SABIO-RK website returned 519 entries, which is a less amount of information compared to BRENDA. The SABIO-RK website offers an advance search feature that allows filtering by various conditions, such as "organisms: \textit{Homo sapiens}", which narrows the results to 168 entries related to human CES data. Additionally, it is possible to filter the results to include only data from recombinant CES enzyme (82 entries). However, like BRENDA database, SABIO-RK also has the limitation of not separating kinetic parameters by CES isoforms. In conclusion, while SABIO-RK contains less information than BRENDA, it offers better organization and more intuitive filtering options, making it easier to retrieve relevant data.

\subsection{CES structural templates}

At the beginning of any computational molecular modeling substrate-enzyme complex study, the information regarding the structure of both, the enzyme and the substrate, are necessary. The structures of the ligands are relatively easy retrieved from different biological databases, such as those mentioned in the previous section. On the other hand, obtaining the structures of the macromolecules (enzymes) can be more challenging. This is because enzyme structures are often distributed across different sources and databases. In this section, the availability structures of CES1 and CES2 enzymes will be discussed.


\subsubsection{CES1 structures}

Since 2003, different crystallographic structures of CES1 have been reported, as summarized in Table~\ref{table:CES1_structures}. The first published structures involved CES1 complexes with ligands from several categories, including metabolites of abuse drugs like cocaine and heroin (homatropine and naloxone methiodide, respectively)\cite{Bencharit2003}, therapeutic drugs (tacrine, tamoxifen and mevastatin)\cite{Bencharit2003a, Fleming2005}, and endogenous substrates (cholate, taurocholate and coenzyme A) \cite{Bencharit2006} (Table~\ref{table:CES1_structures}). Notably, the crystal structure of CES1-tamoxifen complex revealed that tacrine interacts in the catalytic binding site of CES1 in four different binding modes\cite{Bencharit2003a}. These findings highlights the promiscuous nature of CES1, suggesting that its ability to hydrolyze a variety of substrates is driven by its capacity to interact with these ligands in multiple conformations\cite{Bencharit2003a}.

\begin{center}
	\textbf{Table~\ref{table:CES1_structures}}
\end{center}

Different types of covalent ligands have also been studied through crystallographic structures. The first group to be investigated included the organophosphorus nerve agents soman, tabun and cyclosarin \cite{Fleming2007,Hemmert2010}. More recently, attention shifted to a second family focused on the covalent binding mechanism of serine-selective electrophilic warheads. Two crystal structures were obtained showing CES1 covalently bound to 2,2,2-trifluoroacetophenone derivatives at the catalytic serine (Table~\ref{table:CES1_structures})\cite{Gai2024}. Additionally, there are crystallographic structures of CES1 in the absence of substrates (\textit{apo} form)\cite{Greenblatt2012,ArenadeSouza2015,Su2023}. One of these structures (PDB code: 5A7F) exhibited the highest resolution of any reported CES1 crystal structure to date (1.86~\AA).

For conducting molecular modeling studies focused on substrate selectivity between CES isoforms, the optimal approach is to use the CES1 structure with the highest possible resolution and complexed with non-covalent ligands or in a substrate-free state. On the other hand, it is not advisable to use the crystal structures of CES1 bound to covalent ligands. This is because the catalytic residues, along with other residues in the catalytic binding site, may be biased
to conformations favoring covalent inhibitors rather than modeling the intrinsic substrate recognition associated to the \textit{apo} form of the enzyme.


\subsubsection{CES2 structures}

The crystallographic structure of CES2 remains unresolved to this day. Previous studies indicated that CES are glycoproteins and in some cases, like in CES2, the heterogeneity of N-glycan structures and conformations at the surface of proteins generate higher difficulty for the production of non-glycosylated CES2 for 
crystallization and diffraction studies\cite{Alves2016}. As a result, homology modeling is necessary to obtain the three-dimensional structure of this CES isoform in order to model the binding mode of substrates in the catalytic site of the enzyme.  Different research publications have reported distinct strategies for
generating the homology model structure of CES2. One of the most common methodologies involves using tools provided by the Swiss Institute of Bioinformatics server (www.expasy.org/). In early publications, the process of generating the CES2 model followed a two-steps procedure: first, the amino acid sequence of human CES2 was retrieved from the Swiss-Prot database, and then submitted to the Swiss-Model\cite{Waterhouse2018} server for fully automated protein structure homology modeling\cite{Vistoli2010a,Zou2016,Song2019}. Over time, this methodology became more automated, allowing users to directly download pre-existing CES2 models created by other researchers via the Swiss-Model server\cite{Qian2021,Lv2025,Ribone2025}.

The second approach involves generating the CES2 structure using homology modeling software. Several studies have reported the use of the open-source \textit{Modeler} software\cite{Webb2017} to generate the homology model of CES2 and perform subsequent molecular modeling studies on this isoform\cite{Wang2018,Choudhary2019}.

In recent years, \textit{AlphaFold}, a powerful neural network-based model
methodology has been developed for predicting the three-dimensional 
structures of proteins\cite{Jumper2021}. Since its beginning in 2021, 
this machine learning approach has generated more than 200 million protein 
structures, all freely available for download from the AlphaFold database 
(alphafold.ebi.ac.uk). To the best of our knowledge, despite its accessibility
and remarkable accuracy, an AlphaFold-predicted human CES2 structure has not yet
been utilized for the molecular modeling of this enzyme with any substrate.

\subsection{Molecular modeling methods}

To analyzed the results from experimental enzymology studies of the CES-substrate
complexes, a series of molecular modeling methodologies can be employed. These 
computational techniques were selected to investigate, at an atomistic level,
the kinetic parameters of different substrates for both CES isoforms. This section
is organized in two subsections: 1) Modeling of substrate:enzyme recognition 
to study the affinity constant (K\textsubscript{M}) and 2) Modeling of 
substrate:enzyme reactivity employed to examined the catalytic constant
(\textit{k\textsubscript{cat}}).


\subsubsection{Modeling of substrate:enzyme recognition}

Typically, the study of the recognition and binding of CES substrates begins with a molecular docking procedure, aimed to identify substrate conformations that yield the lowest-energy interactions with residues in the CES catalytic binding site. In a subsequent stage, the CES-substrate complex is subjected to molecular dynamics (MD) simulation to characterize its dynamic behavior and stability in an explicit aqueous environment and at physiological temperature. Finally, a free-energy interaction  analyses are performed based on the MD trajectories obtained for the CES-substrate complex. A small subset of studies incorporate all three of these molecular modeling methods in the investigation of CES-substrate 
complexes. Although a few studies incorporate all three of these molecular modeling approaches when investigating CES–substrate interactions, most rely solely on molecular docking to identify the optimal substrate conformation within the CES catalytic binding site.

Early molecular docking studies designed to explored the selectivity of CES1 and CES2 were conducted by the group of Vistolli et al. \cite{Vistoli2010,Vistoli2010a}. In these works, 40 known CES substrates were subjected to molecular docking protocols using two crystallographic structures of CES1 (PDB codes: 1MX9 and 1YAJ) and a homology modeling structure of CES2. Different scoring functions were calculated and correlated with the experimentally reported K\textsubscript{M} values to develop predictive models of substrate affinity toward both CES isoforms. In both cases, the results revealed a strong correlation between K\textsubscript{M} and the calculated lipophilic interaction scores, highlighting the central role of hydrophobic interactions,
primarily attributable to the abundance of apolar residues within the catalytic binding site\cite{Vistoli2010,Vistoli2010a}.


A combination of molecular docking and MD simulations have
been employed to assist in the design of selective inhibitors of human 
CES2. These studies were focused on the design, synthesis and 
structure-activity relationship of glycyrrhetinic acid and benzofuranone 
derivatives (Figure~\ref{fig:CES2_substrates})\cite{Zou2016,Yang2023}. In the first work,
molecular docking was performed using the most active and selective 
glycyrrhetinic acid derivative against both CES isoforms to
elucidate its 1000-fold preference for CES2 over CES1\cite{Zou2016}.
The results indicated that this derivative displayed a greater number of 
hydrogen bond interactions with residues in the CES2 catalytic binding site 
residues than in CES1. The authors concluded that the high hydrogen
bond interaction of this derivative with the essential catalytic CES2
residue Ser228 totally block the recognition and binding of any CES2 
substrate\cite{Zou2016}. In the second study, a similar molecular 
docking analysis was performed for the most active and selective 
benzofuranone derivative against both isoforms. The results were consistent:
the inhibitor exhibited more favorable interactions and a higher number binding
site contacts with CES2 residues compared to CES1\cite{Yang2023}. Furthermore,
MD simulation and free-energy decomposition analyses of the CES2-inhibitor complex
revealed that the complex remain stable over 50 ns of simulation 
and that the hydrophobic interactions played a major role in the binding 
of the inhibitor to CES2, with a significant contribution from hydrophilic 
residues such is the case of the catalytic Ser228\cite{Yang2023}. As shown in 
Figure~\ref{fig4}, both selective CES2 inhibitors share the presence of a 
carboxylic acid group in their molecular structures, indicating that further 
analysis of this structural feature could orient the design of novel and 
more selective CES2 inhibitors.


In a more recent study\cite{Ribone2025}, molecular docking, MD simulation
and free-energy interaction decomposition analyses were performed on two 
families of previosly studied substrate families: five \textit{p}-nitrophenyl 
ester derivatives\cite{Hatfield2010} and two pyrethroid stereoisomers\cite{Nishi2006}, 
using both CES1 and CES2. Consistent with 
earlier findings, hydrophobic interactions were found to contribute 
substantially to the CES-ligand free-energy of interaction, correlating well
with experimentally determined affinity constants (K\textsubscript{M})\cite{Ribone2025}. 
An additional key observation was that the higher binding cavity volume of CES2, relative to CES1,
enabled one pyrethroid stereoisomer to adopt a distinct interaction pattern and
thereby maintain a higher affinity (lower K\textsubscript{M}) for CES2 
compared to the other stereoisomer. Overall, the substrates analyzed in this work formed the most 
favorable interaction patterns with residues in the respective CES binding sites, positioning
in every case the oxygen of the carbonyl ester moiety in the oxyanion hole thus generating a
conformation close to the catalytic serine residue, resulting in the highest attainable affinity
for each CES isoform\cite{Ribone2025}.

\subsubsection{Modeling of substrate:enzyme reactivity}

The catalytic constant determined in enzyme-catalyzed hydrolytic
reaction requires specialized computational methodologies 
to analyzed the free-energy of activation associated with this
assay. The most commonly employed approach for this purpose is hybrid 
QM/MM simulation. In this method, the ligand and the catalytic residues
directly involved in the hydrolysis reaction are modeled using quantum mechanics
(QM), while the remaining enzyme residues are described with a 
molecular mechanics (MM) force field. The reaction hydrolysis is modeled
using an umbrella sampling approach, an enhanced sampling technique, to 
overcome energy barriers by restraining the system to different points 
along a reaction coordinate in separate "windows". By combining the results
from these biased simulations, the method generates an unbiased potential
of mean force (PMF) or free energy profile, providing a detailed picture of 
the reaction pathway, states, and transition energies.

There are relatively few reported studies employing hybrid QM/MM
simulations for substrates hydrolyzed by human CES, likely
because of the greater complexity of this methodology compared with
the molecular modeling methodologies described in the previous section.
Given the importance of human CES in the metabolism of abuse drugs,
several works have investigated the hydrolysis mechanism of cocaine by
CES1 and CES2. The first study reported the hydrolysis mechanism of 
cocaine by CES1 using hybrid QM/MM simulation alongside the experimental 
determination of the corresponding kinetic parameters (K\textsubscript{M}
and \textit{k\textsubscript{cat}})\cite{Yao2018}. The QM region was
parameterized with the semi-empirical SCC-DFTB method, while the MM
region with CHARMM27 force field. The reaction was modeled using an umbrella
sampling approach, where the reaction coordinate (RC) was defined as a
linear combination of two covalent bond distances: one forming and one
breaking. The simulations indicated that CES1 catalyzes the hydrolysis
of cocaine to benzoylecgonine and methanol (Figure\ref{fig1}) through a single-step 
acylation reaction followed by a single-step deacylation stage, each showing a 
distinct transition state (TS). The combined free-energy profile  of both 
reaction steps revealed that the acylation TS is the rate-limiting 
reaction, with a free energy barrier of 20.1 kcal/mol. 
This result is in close agreement with the experimental free-energy barrier,
derived from the catalytic constant, which was 21.5 kcal/mol, demonstrating 
the high accuracy of the hybrid QM/MM methodology employed in this study.\cite{Yao2018}.

A second article reported a study similar to the previous one,
exploring the hydrolysis of cocaine by CES2 to produce ecgonine 
methyl ester and benzoic acid (Figure~\ref{fig1}). In this study,
the QM region was parameterized using the semi-empirical PM6 method, and 
the MM area was modeled with the ff99SB force field. An umbrella
sampling approach was again employed to follow the two-steps hydrolysis
reaction, in which two covalent bond distances,
one bond forming and one bond breaking, were used as reaction coordinates.
The resulting 2D-PMF profiles for the full catalytic cycle showed
that each reaction stage proceeds through a tetrahedral intermediate
structure and involves two transition states. Among these, TS\textsubscript{4}
(associated with the formation of benzoic acid) was identified as the rate-limiting
step of cocaine hydrolysis by the CES2, displaying an activation free-energy
barrier of 19.5 kcal/mol. This lower energetic barrier respect to 
CES1 is consistent with previous experimental findings that reported 
a higher turnover number for CES2 than for CES1.

In the previous section, a study involving two families of substrates,
\textit{p}-nitrophenyl esters and pyrethroid stereoisomers, in 
complex with CES1 and CES2 was described. In this work,
two \textit{p}-nitrophenyl ester derivatives and two pyrethroid
stereoisomers were subjected to hybrid QM/MM simulation to analyze
their hydrolysis by both CES isoforms\cite{Ribone2025}. The
QM/MM simulation protocol followed was similar to the earlier study
of cocaine with CES1\cite{Yao2018}; the QM region was parameterized 
using the semi-empirical SCC-DFTB3 method, while the MM region was modeled
with Amber20 force field (ff14SB). As in the previous example, the study
employed an umbrella sampling approach; however, in this case, four 
covalent bond distances, two bonds forming and two bonds breaking 
(Figure~\ref{fig5}), were combined linearly using a linear combination 
of distances (LCOD) to define the reaction coordinate (RC).

Across all modeled reactions, substrate hydrolysis by both CES isoforms
proceeded through a concerted single-step acylation stage followed by 
another single-step deacylation stage, each involving one transition state (TS).
Analysis of the results showed that the rate-limiting step for each 
substrate corresponded to the TS associated with the highest 
molecular steric hindrance encountered during the course of 
the hydrolysis, acylation or deacylation stage\cite{Ribone2025}. 
The authors concluded that the CES selectivity is not solely determined
by the molecular size of the alkyl or acyl groups of the substrates,
but instead arises from a more complex scenario governed by the
initial conformation of the ligand within the CES binding site\cite{Ribone2025}.

Overall, all studies discussed in this section showed a good 
correlation between the experimental catalytic constant
(\textit{k\textsubscript{cat}}) and the calculated free-energy of
activation from the rate-limiting TS reaction stage, demonstrating
the reliability of the hybrid QM/MM methodology. This was observed
despite the differences in the proposed hydrolysis mechanisms, 
which involved either two or four transition-state structures.

\section{Part III: Combination of experimental and computational enzymology} 

This final section of the article is devoted to studies that combined
experimental and computational methodologies to investigate
the hydrolysis properties of various substrates by both
CES isoforms.

In the first study, the hydrolysis of fenofibrate by CES1 and CES2 was 
examined to identify the primary enzyme responsible
for its metabolization in humans\cite{Li2023}. The kinetic parameters were 
determined by measuring the formation rate of fenofibric acid 
(the hydrolytic metabolite of fenofibrate) through HPLC-UV in the presence
of human microsoms and recombinant CES1 and CES2. These results exhibited 
that the affinity constant was slightly higher for CES1, but the major difference
was observed in the catalytic constant, with CES1 exhibiting a substantially
faster reaction rate than CES2\cite{Li2023}. 

The authors performed a molecular docking study of fenofibrate within 
the binding site of both CES isoforms. The docking results showed that 
fenofibrate adopted a more favorable binding pose and received a better 
docking score into the catalytic binding site of CES1 than CES2, consistent
with the experimental affinity results. Hybrid QM/MM
simulations were not performed in this publication\cite{Li2023}. This
study provides useful insight into the factors underlying the pronounced 
differences in fenofibrate hydrolysis between the two CES isoforms.

The second study, reported the design and CES selectivity of a family
of four fluorescent substrates derived from a naphthalimide scaffold
\cite{Jia2021}. Enzymatic assay was monitored by measuring the 
emission intensity of the hydrolysis product at 520 nm (excitation 450 
nm) after incubation with CES1 and CES2. Only the derivatives 
containing amide and carbamate moieties (NIC-1 and NIC-2, 
figure~\ref{fig:NIC_hydrolysis}) underwent specific hydrolysis in the presence of CES2. In contrast, the other two carbamates in which the nitrogen was fully
substituted with carbon atoms (NIC-3 and NIC-4, figure~\ref{fig:NIC_hydrolysis}),
exhibited inhibitory effects of both CES isoforms.

To investigate the shift in activity from CES substrate to inhibitor,
the binding properties of NIC-4 toward CES1 were analyzed using molecular
modeling methods with CES1 and compared with the known substrate 
\textit{p}-nitrophenyl acetate (pNPA)\cite{Jia2021}. Molecular docking 
followed by MD simulations, revealed that both NIC-4 and pNPA displayed similar 
interaction patterns with the residues of the CES1 binding site.
In addition, a steered MD simulation was performed in which the structures 
and the binding site residues were parameterized with SCC-DFTB method, while
the remaining part of the enzyme with Amber ff14SB force field. The reaction
coordinate was defined as the distance between the catalytic serine oxygen and
the carbonyl carbon of the substrate, representing the nucleophilic attack 
in the hydrolysis reaction. During the simulation of pNPA hydrolysis by CES1,
the proton transfer from the catalytic serine to the nitrogen atom of the 
catalytic histidine occurred spontaneously, following the normal acylation
mechanism with an estimated energetic barrier of 20 kcal/mol. 
In contrast, during the simulation of NIC-4 with CES1, this proton transfer 
did not occur, because the methyl group positioned near the potential 
nucleophilic center of NIC-4 (Figure~\ref{fig6}) sterically blocked the 
transfer pathway, forcing the serine hydroxyl group to orient away from the
histidine imidazole. The authors concluded that NIC-4 mimics the interaction
pattern of the classic substrate to benefit enzyme binding but hinders the 
necessary proton transfer process by methyl substitution of nitrogen in
the carbamate, thus preventing the nucleophilic attack from happening\cite{Jia2021}.

This last work is a clear example that integration of experimental and molecular 
modeling results analysis are extremely neccessary to explain with higher accuracy
the mechanism behind the substrate CES selectivity and also to assist in the 
design to selective CES inhibitors.

\section{Conclusions}

The objective of this article was, in first place, to highlight the importance of combining experimental and computational techniques to elucidate substrate selectivity between the two major human CES isoforms: CES1 and CES2. A second
objective derive from the last one, giving different tools to researchers specialized in experimental enzymology to learn more about the different molecular modeling methodologies that can assist for the explanation of their
enzymatic obtained results.

The first section showed that separation chromatographic methodologies couple with
quantification techniques (UV and fluorescent spectrophotomety or mass spectrometry) are widely used by different research groups for the experimental determination of CES kinetic parameters (K\textsubscript{M} and \textit{k\textsubscript{cat}}). The corresponding kinetic data and experimental methodologies can be found in several freely accessible enzymatic databases, including BRENDA and SABIO-RK.

The second section summarized the computational strategies employed to model substrate : CES recognition and affinity, through molecular docking, MD simulation and free-energy of binding analysis, and catalytic turnover constant, through hybrid QM/MM simulation, for the studied the substrate : CES hydrolysis. These computational methodologies demonstrated high accuracy and provided results consistent with experimentally reported data.

The advantages of integrating experimental and computational approaches became
clear in the examples discussed in the final section, where the combined use of
both methodologies enabled the elucidation of structural factors underlying 
substrate selectivity toward specific CES isoforms.

This collaborative strategy should be more widely adopted to accelerate advances
in understanding CES substrate selectivity for future compounds. 
Given the biological relevance of CES-mediated catalysis, further investigation
into the structural determinants governing isoform specific substrate recognition
remains a significant scientific challenge, offering valuable insights for the 
rational design of selective CES inhibitors, prodrug designed with an optimized 
bioactivation location and the design of selective fluorescent biological probes.
It is hoped that this article will serve as a useful resource to support continued
exploration in this field.

\newpage

\bibliographystyle{ieeetr}
\bibliography{JoX-2025.bib}{}

\newpage

\begin{figure}[!h]
	\centering
		\includegraphics[scale=0.4]{Figure5.png}
		\caption{Acylation and deacylation reaction pathways for CES1 and CES2 catalyzed hydrolysis of the ester ligands. \textit{p}-nitrophenyl ester derivatives were used as example. Reactant state (RS), first transition state (TS1), intermediate (INT), second transition state (TS2) and product state (PS).}
		\label{fig:CES_mechanism}
\end{figure}


\newpage

\begin{figure}[!h]
	\centering
		\includegraphics[scale=0.4]{Figure1.png}
		\caption{Hydrolysis of \textit{p}-nitrophenyl acetate (pNPA) by CES to produce \textit{p}-nitrophenol.}
		\label{fig:pNPA_hydrolysis}
\end{figure}

\newpage

\begin{figure}[!h]
	\centering
		\includegraphics[scale=0.4]{Figure2a.png}
		\caption{Hydrolysis of fluorescein diacetate to produced fluorescein as fluorescent substrates..}
		\label{fig:fluorescein_hydrolysis}
\end{figure}

\newpage

\begin{figure}[!h]
	\centering
		\includegraphics[scale=0.4]{Figure2b.png}
		\caption{Hydrolysis of BODIPY ester to released	a BODIPY as fluorescent substrates.}
		\label{fig:BODIPY_hydrolysis}
\end{figure}

\newpage

 \begin{figure}[!h]
 	\centering
 		\includegraphics[scale=0.4]{Figure3.png}
 		\caption{Different cocaine metabolic pathway from CES1 and CES2  experimental hydrolysis studies.}
 		\label{fig:cocaine_hydrolysis}
 \end{figure}

\newpage


\begin{figure}[!h]
	\centering
	\begin{subfigure}[b]{0.40\textwidth}
		\centering        
		\includegraphics[scale=0.7]{Figure4a.png}
		\caption{\centering ...}
		\label{fig:GEN-general}
	\end{subfigure}
	%[FREDY_rev1: I think that this general structure of GEN should be included in the introduction section, and maybe a table indicating the structure of the four conegeners]
	
	\begin{subfigure}[b]{0.40\textwidth}
		\centering        
		\includegraphics[scale=0.7]{Figure4b.png}
		\caption{\centering ...}
		\label{fig:GEN-specific}
	\end{subfigure}
	\caption{Molecular structures of CES2 selective inhibitors. (\textbf{a}) Glycyrrhetinic acid derivative. (\textbf{b})  Benzofuranone derivative}
	\label{fig:CES2_substrates}
\end{figure}


\newpage

\begin{figure}[!h]
	\centering
	\includegraphics[scale=0.4]{Figure6.png}
	\caption{Structure of the fluorescent substrates derivatives of 	naphthalimide (NIC)\cite{Jia2021}.}
	\label{fig:NIC_hydrolysis}
\end{figure}


\newpage

\begin{table}
\caption{Reported substrates with their respective enzymatic source and analytical methods used for the exploration of CES kinetic parameters.}
\label{table:data}
	
\begin{tabular}{|p{4cm}|p{2cm}|p{3cm}|p{3cm}|p{2.5cm}|p{1.5cm}|}
	\hline
	%\textbf{Substrates} &   &  &  &  &  \\
	\textbf{Substrates}	& \multicolumn{3}{c}{\textbf{Enzymatic sources}} & \textbf{Analytical method} & \textbf{Ref.}\\
	\hline
			& \textit{Tissue} & \textit{H. miocrosomes} & \textit{Recomb. enzyme} &\\
			\hline 
			Irinotecan	& Liver	& - & - & HPLC-FL & \cite{Humerickhouse2000,Sanghani2004}\\
					Methylphenidate	& Liver	& - & CES1/CES2 & LC/MS & \cite{Sun2004}\\
					Oseltamivir	& -	& HLM/HIM & - & HPLC-UV & \cite{Shi2006,Fukami2015}\\
					Temocapril	& -	& HLM/HIM & CES1/CES2 & HPLC-UV & \cite{Imai2006,Fukami2015}\\
					Aspirin	& -	& HLM/HIM & CES1/CES2 & HPLC-UV & \cite{Imai2006,Tang2006}\\
					Clopidogrel	& -	& HLM/HIM & CES1/CES2 & HPLC-UV & \cite{Tang2006,Fukami2015}\\
					Flutamide & -	& HLM/HIM & CES1/CES2 & HPLC-UV & \cite{Watanabe2009,Kobayashi2012}\\
					Pyrethroids & -	& - & CES1/CES2 & FL & \cite{Nishi2006,Ross2006}\\
					Fluorescein diacetate & -	& HLM/HIM & CES1/CES2 & FL & \cite{Wang2011}\\
					Plasugrel & -	& - & CES1/CES2 & LC/MS & \cite{Williams2008}\\
					Heroin & -	& - & CES1/CES2 & HPLC-UV & \cite{Hatfield2010}\\
					Cocaine & -	& - & CES1/CES2 & HPLC-UV & \cite{Hatfield2010,Yao2018}\\
					Oxybutynin & -	& HLM & CES1/CES2 & LC/MS & \cite{Sato2012}\\
					Prilocaine & -	& HLM & CES1/CES2 & HPLC-UV & \cite{Higuchi2013}\\
					Lidocaine & -	& HLM & CES1/CES2 & HPLC-UV & \cite{Higuchi2013}\\
					Clofibrate & -	& - & CES1/CES2 & HPLC-UV & \cite{Fukami2015}\\
					Fenofibrate & -	& HLM/HIM & CES1/CES2 & HPLC-UV & \cite{Fukami2015,Li2023}\\
					Imidapril & -	& - & CES1/CES2 & HPLC-UV & \cite{Fukami2015}\\
					Enalapril & -	& HLM & CES1/CES2 & LC/MS & \cite{Thomsen2014}\\
					Sacubitril & Liver & - & CES1/CES2 & LC/MS & \cite{Shi2016}\\
					Anordrin & - & HLM/HIM & CES1/CES2 & LC/MS & \cite{Jiang2018}\\	
	\hline
\end{tabular}
\end{table}

\newpage

\begin{table}
	\caption{Information related to CES1 reported crystallographic structures.}
	\label{table:CES1_structures}
	\begin{tabular}{|p{3cm}|p{3cm}|p{3cm}|p{3cm}|p{2.5cm}|}
		\hline	
		\textbf{Substrates}	& \textbf{Classification} & \textbf{Resolution (\AA)} & \textbf{PDB code} &  \textbf{Reference}\\
		\hline
		Homatropine	& M.L. & 2.80 & 1MX5 & \cite{Bencharit2003}\\
		Naloxone methiodide	& M.L. &2.90 & 1MX9 & \cite{Bencharit2003}\\
		Tacrine	& Drug &2.40 & 1MX1 & \cite{Bencharit2003a}\\
		Tamoxifen & Drug &3.20 & 1YA4 & \cite{Fleming2005}\\
		Mevastatin & Drug &3.00 & 1YA8 & \cite{Fleming2005}\\
		Ethylacetate&M.L.& 3.00& 1YAH & \cite{Fleming2005}\\
		Benzil & M.L.&3.20& 1YAJ & \cite{Fleming2005}\\
		Cholate/Palmitate &E.S.& 3.00& 2DQY & \cite{Bencharit2006}\\
		CoenzymeA& E.S.&2.00 & 2H7C& \cite{Bencharit2006}\\
		CoenzymeA/Palmitate& E.S.&2.80 & 2DQZ& \cite{Bencharit2006}\\
		Taurocholate & E.S.&3.20& 2DR0 & \cite{Bencharit2006}\\
		Soman & N.A.& 2.70& 2HRQ & \cite{Fleming2007}\\
		Tabun & N.A.& 2.70& 2HRR & \cite{Fleming2007}\\
		Cyclosarin & N.A.& 3.10 & 3K9B & \cite{Hemmert2010}\\
		- & -& 2.20 & 4AB1 & \cite{Greenblatt2012}\\
		- & -& 1.86 & 5A7F & \cite{ArenadeSouza2015}\\
		- & -& 2.67 & 8EOR & \cite{Su2023}\\
		F-3 & C.I.& 1.83 & 9KWL &\cite{Gai2024}\\
		F-4 & C.I.& 1.89 & 9KWM &\cite{Gai2024}\\
		\hline
		
		
		
	\end{tabular}
\end{table}
	
\end{document}